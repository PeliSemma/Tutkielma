% --- Template for thesis / report with tktltiki2 class ---
% 
% last updated 2013/02/15 for tkltiki2 v1.02

\documentclass[finnish]{tktltiki2}

% tktltiki2 automatically loads babel, so you can simply
% give the language parameter (e.g. finnish, swedish, english, british) as
% a parameter for the class: \documentclass[finnish]{tktltiki2}.
% The information on title and abstract is generated automatically depending on
% the language, see below if you need to change any of these manually.
% 
% Class options:
% - grading                 -- Print labels for grading information on the front page.
% - disablelastpagecounter  -- Disables the automatic generation of page number information
%                              in the abstract. See also \numberofpagesinformation{} command below.
%
% The class also respects the following options of article class:
%   10pt, 11pt, 12pt, final, draft, oneside, twoside,
%   openright, openany, onecolumn, twocolumn, leqno, fleqn
%
% The default font size is 11pt. The paper size used is A4, other sizes are not supported.
%
% rubber: module pdftex

% --- General packages ---

\usepackage[utf8]{inputenc}
\usepackage[T1]{fontenc}
\usepackage{lmodern}
\usepackage{microtype}
\usepackage{amsfonts,amsmath,amssymb,amsthm,booktabs,color,enumitem,graphicx}
\usepackage[pdftex,hidelinks]{hyperref}
\usepackage{apacite}
\usepackage{mdframed}
\usepackage{minted}
\usepackage{caption}
\usepackage{listings}

\newenvironment{caplab}[2]
{%
  \renewcommand\listingscaption{Koodi}%
  \vspace{-5pt}%
  \begin{listing}[H]%
  \caption{#2}%
  \label{#1}%
}
{%
  \vspace{7pt}%
  \end{listing}%
  \vspace{-5pt}%
}

\newenvironment{code}[1]
{%
  \VerbatimEnvironment
  \begin{mdframed}%
  \begin{minted}{#1}%
}
{%
  \end{minted}%
  \end{mdframed}%
  \vspace{-15pt}%
}



% Automatically set the PDF metadata fields
\makeatletter
\AtBeginDocument{\hypersetup{pdftitle = {\@title}, pdfauthor = {\@author}}}
\makeatother

% --- Language-related settings ---
%
% these should be modified according to your language

% babelbib for non-english bibliography using bibtex
\usepackage[fixlanguage]{babelbib}
\selectbiblanguage{finnish}

% add bibliography to the table of contents
\usepackage[nottoc]{tocbibind}
% tocbibind renames the bibliography, use the following to change it back
\settocbibname{Lähteet}

% --- Theorem environment definitions ---

\newtheorem{lau}{Lause}
\newtheorem{lem}[lau]{Lemma}
\newtheorem{kor}[lau]{Korollaari}

\theoremstyle{definition}
\newtheorem{maar}[lau]{Määritelmä}
\newtheorem{ong}{Ongelma}
\newtheorem{alg}[lau]{Algoritmi}
\newtheorem{esim}[lau]{Esimerkki}

\theoremstyle{remark}
\newtheorem*{huom}{Huomautus}


% --- tktltiki2 options ---
%
% The following commands define the information used to generate title and
% abstract pages. The following entries should be always specified:

\title{Rust peliohjelmoinnissa}
\author{Victor Bankowski, Antti Karjalainen ja Janne Pulkkinen}
\date{\today}
\level{Seminaariraportti}
\abstract{Tiivistelmä.}

% The following can be used to specify keywords and classification of the paper:

\keywords{avainsana 1, avainsana 2, avainsana 3}

% classification according to ACM Computing Classification System (http://www.acm.org/about/class/)
% This is probably mostly relevant for computer scientists
% uncomment the following; contents of \classification will be printed under the abstract with a title
% "ACM Computing Classification System (CCS):"
% \classification{}

% If the automatic page number counting is not working as desired in your case,
% uncomment the following to manually set the number of pages displayed in the abstract page:
%
% \numberofpagesinformation{16 sivua + 10 sivua liitteissä}
%
% If you are not a computer scientist, you will want to uncomment the following by hand and specify
% your department, faculty and subject by hand:
%
% \faculty{Matemaattis-luonnontieteellinen}
% \department{Tietojenkäsittelytieteen laitos}
% \subject{Tietojenkäsittelytiede}
%
% If you are not from the University of Helsinki, then you will most likely want to set these also:
%
% \university{Helsingin Yliopisto}
% \universitylong{HELSINGIN YLIOPISTO --- HELSINGFORS UNIVERSITET --- UNIVERSITY OF HELSINKI} % displayed on the top of the abstract page
% \city{Helsinki}
%


\begin{document}

% --- Front matter ---

\frontmatter      % roman page numbering for front matter

\maketitle        % title page
\makeabstract     % abstract page

\tableofcontents  % table of contents

% --- Main matter ---

\mainmatter       % clear page, start arabic page numbering

\section{Johdanto}

% Write some science here.
Rust on käännettävä ohjelmointikieli, jonka kehitystä tukee Mozilla-säätiö \cite{servo}. Mozilla käyttää kieltä uuden rinnakkaisuutta hyödyntävän Servo -internet-selainmoottorin ohjelmointiin [ja lisäksi käytetään missä?]. Käännettävänä ohjelmointikielenä C ja C++ -kielten tavoin Rust mahdollistaa suorituskykyä ja hallittua muistin käyttöä vaativien sovellusten kehittämisen esimerkiksi sulautetuissa järjestelmissä. Edellä mainituista kielistä poiketen Rust kuitenkin estää yleisiä C-kielissä esiintyviä muistinhallintaa ja kilpatilanteita koskevia ongelmia, mahdollistaen kuitenkin vastaavan suorituskyvyn ajettavassa ohjelmassa. Rust ratkaisee nämä ongelmat käyttämällä muistinhallinnassa omistajuuden (“ownership”) ja lainaamisen (“borrowing”) käsitteitä. Tämä estää mahdolliset virhetilanteet jo ohjelman käännösvaiheessa vaatimatta virtuaalikoneen, kääntäjän tai tulkin käyttöä ohjelman suorituksen aikana.

\section{Historia}

Rust-kielen kehitys alkoi vuonna 2006 Graydon Hoaren sivuprojektina, jollaisena se jatkui yli kolmen vuoden ajan\footnote{https://www.rust-lang.org/en-US/faq.html}. Mozilla-säätiö osallistui kehitykseen ensimmäisen kerran vuonna 2009 ja on tukenut ohjelmointikielen kehitystä siitä lähtien. Nykyisin kieltä kehittävä ryhmä -- \textit{The Rust Team} -- jakautuu osaryhmiin, jotka vastaavat kielen eri osa-alueista. Osa-alueisiin kuuluvat esimerkiksi kääntäjän kehittäminen, kielen ominaisuuksien suunnittelu ja dokumentaatio.

Suurimpiin Rustia käyttäviin projekteihin kuuluu Mozillan kehittämä Servo -web-selainmoottori. Sen tavoitteisiin kuuluu sivun piirtämisen, HTML-datan parsimisen ja muiden web-selaimen piiriin kuuluvien tehtävien rinnakkaistaminen\footnote{https://hacks.mozilla.org/2017/08/inside-a-super-fast-css-engine-quantum-css-aka-stylo/}. Servo-projektiin kuuluva CSS-moottori Stylo on otettu käyttöön Mozilla Firefox -selaimen uusissa kehitysversioissa\footnote{https://blog.mozilla.org/blog/2017/09/26/firefox-quantum-beta-developer-edition/}.

Muihin Rust-kieltä hyödyntäviin organisaatioihin kuuluu muun muassa Dropbox ja Canonical\footnote{https://www.rust-lang.org/en-US/friends.html}.

\section{Perusteet}
\begin{caplab}{helloworld}{Tulostaa "Hei maailma"}
\begin{code}{rust}
fn main() {
    println!("Hei maailma");
}
\end{code}
\begin{code}{bash}
$ cargo run
Hei maailma
\end{code}
%$
\end{caplab}


Ohjelmointikieleen tutustuessa on tapana kirjoittaa klassinen "Hei maailma" -ohjelma joka tulostaa kyseisen lauseen. Koodi \ref{helloworld} on kyseisen ohjelman Rust toteutus. Kyseinen toteutus ei poikkea juurikaan muiden proseduraalisten kielien toteutuksista. Suurin ero muiden kielien toteutuksiin on se että tulostuskomento on makro. Tulostuskomento on toteutettu makrolla tekstin muotoilun helpottamiseksi. Koodissa \ref{helloworld2} on esimerkki tästä.
\begin{caplab}{helloworld2}{Tulostaa "Hei maailma"}
\begin{code}{rust}
fn main() {
    let s_luku: i32 = 4; // Valittu reilulla nopanheitolla.
    println!("Satunnaislukusi on {}", s_luku);
}
\end{code}
\end{caplab}

Rustissa muuttujat määritellään käyttäen \textbf{let} avainsanaa ja muuttujan tyyppi erotellaan kaksoipistellä. Koodissa \ref{helloworld2} muuttuja nimeltä $s\_luku$ on 32-bittinen etumerkillinen kokonaisluku. Muuttujat oletusarvoisesti eivät ole muokattavissa. Muokattavat muuttujat määritellään käyttäen avainsanayhdistelmää \textbf{let mut}. 

\begin{caplab}{factorialiter}{Funktio joka laskee $n!$ iteratiivisesti.}
\begin{code}{rust}
fn factorial(n: u64) -> u64 {
    let mut result = 1;
    for i in 1..(n + 1) {
        result *= i;
    }
    return result;
}
\end{code}
\end{caplab}

Useimmissa tapauksissa muuttujan tyypin voi jättää merkkaamatta, koska Rust osaa päätellä sen käännösaikana. Kuitenkin funktioiden parametrien ja palautusarvon tyypit täytyy merkata, koska Rustissa ei ole ohjelmanlaajuista tyyppipäättelyä. Funktion palautusarvon tyyppi määritellään nuolen \textbf{->} jälkeen. Koodissa \ref{factorialiter} funktion parametri n ja palautusarvo ovat 64-bittisiä etumerkittömiä kokonaislukuja.

Rustin \textbf{for}-silmukat ovat \textit{for--each}-tyyppisiä, jossa käydään iteraattorin kaikki alkiot läpi. Esimerkiksi koodissa \ref{factorialiter} käydään kaikki välin $[1, n+1)$ kokonaislukuarvot läpi.

Rustissa ei tarvitse käyttää \textbf{return} avainsanaa, jos arvo palautetaan funktion lopussa. Tällöin ei myöskään merkata puolipistettä.

\begin{caplab}{factorialrec}{Funktio joka laskee $n!$ rekursiivisesti.}
\begin{code}{rust}
fn factorial(n: u64) -> u64 {
    if n == 0 {
        1
    } else {
        n * factorial(n-1)
    }
}
\end{code}
\end{caplab}

\subsection{Omistajuus}


\section{Seuraava juttu}
% --- References ---
%
% bibtex is used to generate the bibliography. The babplain style
% will generate numeric references (e.g. [1]) appropriate for theoretical
% computer science. If you need alphanumeric references (e.g [Tur90]), use
%
% \bibliographystyle{babalpha-lf}
%
% instead.

\bibliographystyle{apacite}
\bibliography{references-fi}


% --- Appendices ---

% uncomment the following

% \newpage
% \appendix
% 
% \section{Esimerkkiliite}

\end{document}
